Today we're going to discuss a way to describe schemes as decomposed into simpler parts. One way to do this is to
use connected components, but there's another notion, called reducibility, which is more general, and which we'll
use more frequently. The idea is that $\Spec k[x,y]/(xy)$, the $x$- and $y$-axes inside $\A_k^2$, is a union of two
$\A_k^1$s, so we want to call it reducible.

\begin{lem}
\label{fNgnil}
Let  $Z\subseteq X$ be an irreducible component and $f$ be a function on $X$ with $f|_Z = 0$. Then there's a
$g\in\Fun(X)$ with $g|_Z \ne 0$ and $f^Ng = 0$ for $N\gg 0$.
\end{lem}
\begin{proof}
For $X$ affine, $Z$ is equivalent data to a prime ideal $\p$ because $Z$ is irreducible, and minimal because $Z$ is
a component, and $f$ must be in $\p$. The localization $A_\p$ is a local ring with the image of $f$ contained in
the maximal ideal, and $A_\p[f^{-1}] = 0$.

In general, if you're localizing at a multiplicative set and obtain zero, then some element of the set is zero. We
localized with respect to $sf^N$ for $N\ge 0$ and $s\not\in\p$, so we conclude that there's some $g\in
A\setminus\p$ (so a global function on $X$ nonvanishing on $Z$), and such that $gf^N = 0$ for some $N$.
\end{proof}
\begin{rem}
Localization has a geometric interpretation. Suppose $X = \Spec A$ and $B = A/\p$, so $B$ is an integral domain.
Let $k$ denote its fraction field; if $Z = \Spec B$, then $\Spec k\inj Z$ is the \term{generic point} of $Z$, in
the sense that in the Zariski topology, its closure contains all other points. This is perhaps a bit bizarre, but
it allows for some useful constructions: the localization $\Spec A_\p$ admits the geometric description of the
intersection\footnote{Well, actually an inverse limit, not an intersection. But in reasonable situations, these are
the same thing.} of all opens $U\subseteq X$ which contain the generic point $\Spec k$.
\end{rem}
\begin{lem}
Let $X$ be an affine Noetherian scheme and $Z\subseteq X$ be an irreducible component. Then there's a function $g$
on $X$ with $g|_Z\ne 0$ and $g|_{X\setminus Z} = 0$.
\end{lem}
\begin{proof}
Writing $X = \Spec A$, $Z$ corresponds to a prime ideal $\p$, necessarily finitely generated because $X$ is
Noetherian: $\p = (f_1,\dotsc,f_r)$. By \cref{fNgnil}, we can choose an $N > 0$ and $g_1,\dotsc,g_r\in\Fun(X)$ such
that $g_if_i^N = 0$ and $g_i|_Z \ne 0$. Letting $g = \prod g_i$, it's nonzero on $Z$, because $Z$ is Spec of a
domain.

We claim $g|_{X\setminus Z} = 0$. This is because $X\setminus Z$ is covered by $D(f_i) = \set{f_i\ne 0}_{i =
1,\dotsc,r}$, so it's enough to see on each $D(f_i)$; since $gf_i^N = 0$, then $g|_{D(f_i)} = 0$.
\end{proof}
\begin{cor}
\label{genericnotcomp}
If $Z\subseteq X$ is an irreducible component, then $X\setminus Z$ isn't dense in $X$.
\end{cor}
\begin{proof}
We can easily reduce to $X$ affine. Then pick a $g$ with $g|_Z\ne 0$ and $g|_{X\setminus Z} = 0$. This means
$\overline{X\setminus Z}\subseteq\set{g = 0}\subsetneq X$, since its complement $D(g)$ is an open subscheme of $X$.
\end{proof}
\begin{cor}
If $X$ is a Noetherian scheme, it has only finitely many irreducible components.
\end{cor}
\begin{proof}
Let $\set{Z_i}_{i\in I}$ be the set of irreducible components of $X$, and let
\begin{equation}
	Y_i\coloneqq \overline{X\setminus (Z_1\many\cup Z_i)}.
\end{equation}
We claim $Z_i\subsetneq Y_i$ but $Z_{i+1}\subseteq Y_i$ --- we'll prove this in just a sec, but assuming the claim
we obtain a decreasing sequence $Y_1\supsetneq Y_2\supsetneq\dotsb$ if closed subschemes, so since $X$ is
Noetherian, $\abs I$ must be finite.

Now the claim. We saw that the generic point of $Z_i$ isn't in $\overline{X\setminus Z_i}$ in
\cref{genericnotcomp}, and by definition $\overline{X\setminus Z_i}\supseteq Y_i$. Since $Z_{i+1}\subseteq Y_i$ and
\begin{equation}
	Z_{i+1}\cap X\setminus\paren{Z_1\many\cup Z_i} \ne\varnothing,
\end{equation}
because the components $Z_i$ are distinct, then by irreducibility, $Z_{i+1}$ is contained in the closure of
$X\setminus\paren{Z_1\many\cup Z_i}$.
\end{proof}
\begin{defn}
The \term{Krull dimension} $\dim X$ of a scheme $X$ is the largest integer $d$ such that there is a chain
$Z_0\subsetneq Z_1\subsetneq\dotsb\subsetneq Z_d$ of irreducible subschemes of $X$. If there is no such integer, we
say the dimension is infinite.
\end{defn}
There's one main result in dimension theory.
\begin{thm}
\label{dimthm}
Let $X$ and $Y$ be finite type, irreducible $k$-schemes. If $f\colon Y\to X$ is a morphism, then there's a dense
open $U\subseteq X$ (equivalently, $U$ is nonempty), such that either $Y\times_X Y = \varnothing$ or for every
field-valued point $x\in U$, $\dim f^{-1}(x) = \dim Y - \dim X$.
\end{thm}
Here $f^{-1}(x)\coloneqq Y\times_X\set x$, as usual. A \term{field-valued point} is data of a field $k$ and a map
$x\colon \Spec k\to U$. We'll prove this next time, and spend the rest of today's lecture on some corollaries.
\begin{cor}
If $X$ is irreducible and $U\subseteq X$ is dense, then $\dim U = \dim X$.
\end{cor}
\begin{cor}
If $X$ is irreducible and $f\colon Y\to X$ is dominant, then $\dim Y\ge\dim X$.
\end{cor}
Recall that dominant means it's injective on the level of functions. This means that when we tensor with the
fraction field of $X$, we get something nonzero, and therefore the fiber over the generic point is nonempty.

There are nice situations in which these theorems aren't true. For example, let $A$ be a discrete valuation ring,
such as $k[t]_{(t)}$ or $k[[t]]$. Then $k\coloneqq A[t^{-1}]$ is the fraction field of $A$, and $\Spec k\inj \Spec
A$ is a nonempty open which is zero-dimensional, but $\Spec A$ is one-dimensional. So we really have to use the
fact that we're finite type over a field.

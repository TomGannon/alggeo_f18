\begin{quote}\textit{
	``I'll stick to the party line.''
}\end{quote}
\begin{lem}
Let $k$ be a field, $A$ be a $k$-algebra, and $f\in A$. If $j\colon U\inj X = \Spec A$ denotes the locus where
$f\ne 0$, then restriction defines an isomorphism $j^*\Omega_X^1\overset\cong\to\Omega_U^1$.
\end{lem}
\begin{proof}
Let $M$ be an $A[f^{-1}]$-module, which is the same thing as a quasicoherent sheaf on $U$. Thinking of $M$ as an
$A$-module (i.e.\ $j^*M$), a derivation $\delta\colon A\to M$ is the same thing as a map $j^*\Omega_X^1\to M$ by
adjunction. Then $\delta$ extends uniquely to a derivation $\widetilde\delta$ on $A[f^{-1}]$, because we know what
it has to be on $f^{-1}$ by the Leibniz rule:
\begin{equation}
	\widetilde\delta(f^{-n}) = -n f^{-n-1}\widetilde\delta(f).
\end{equation}
This is a quick inductive argument: we know $\widetilde\delta(1) = 0$, so
\begin{equation}
	0 = \widetilde\delta(f^{-n}f^n) = f^n\widetilde\delta(f^{-n}) + f^{-n}\widetilde\delta(f^n) =
	f^n\widetilde\delta(f^{-n}) + nf^{-1}\widetilde\delta(f).
\end{equation}
Now, given an arbitrary element $g/f^n\in A[f^{-1}]$, we define
\begin{equation}
	\widetilde\delta\paren{\frac{g}{f^n}} = g(-n)f^{-n-1}\delta(f) + f^{-n}\delta(g).
\end{equation}
It's fairly straightforward to check this is well-defined, and that it gives a derivation.
\end{proof}
As a corollary, we can define $\Omega_X^1$ as a quasicoherent sheaf on any $k$-scheme $X$: using Serre's theorem,
it suffices to describe it on any open affine $j\colon U\inj X$, where it's just $\Omega_U^1$. The above lemma
guarantees this behaves correctly on intersections.
\begin{defn}
\label{smoothdefn}
A $k$-scheme $X$ is \term{smooth} if
\begin{enumerate}
	\item $X$ is locally of finite type over $k$,
	\item $\Omega_X^1$ is a vector bundle, and
	\item for all irreducible components $Z\subseteq X$, the rank of $\Omega_X^1|_Z$ is equal to $\dim Z$.
\end{enumerate}
\end{defn}
\begin{rem}
In practice, $X$ will generally be finite type.
\end{rem}
It is a nontrivial fact that if $X$ is smooth, every irreducible component is a connected component.
\begin{rem}
Our definition of irreducible is slightly more restrictive than the standard definition, which allows things such
as $\Spec k[\e]/(\e^2)$. What we call irreducible is generally called integral. Fortunately, it doesn't make a
difference in \cref{smoothdefn}, though this isn't obvious.
\end{rem}
\begin{comp}{exm}{enumerate}
	\item $\A_k^n$ is smooth, because $\Omega_{\A_k^n}^1$ is free of rank $n$.
	\item $\P_k^n$ is also smooth, because it has a cover by copies of $\A_k^n$, which is smooth.
	\item If $n\ge 2$, then $X = \Spec k[t]/(t^n)$ is not smooth: it's zero-dimensional, but $\Omega_X^1\ne 0$.
	\item Consider the coordinate axes in $\A_k^2$, $X \coloneqq \Spec A$, where $A \coloneqq k[x,y]/(xy)$. This is
	not smooth.  It's one-dimensional, and as an $A$-module,
	\begin{equation}
		\Omega_X^1 = A[\d x, \d y]/(x\ud y + y\ud x).
	\end{equation}
	We have a resolution of this module defining generators and relations:
\begin{equation}
	\xymatrix@C=1.5cm{
		A\ar[r]^-{1\mapsto x\ud y + y\ud x} & A^{\oplus 2}\ar[r]^-{(\d x,\d y)} & \Omega_X^1\ar[r] & 0.
	}
\end{equation}
	Restricting to $(0,0)\in X$, we get
\begin{equation}
	\xymatrix{
		k\ar[r]^-0 & k^{\oplus 2}\ar[r]^-\cong & \Omega_X^1|_{(0,0)}\ar[r] & 0,
	}
\end{equation}
	so here it has rank 2, which is not the dimension of $X$. Therefore $X$ isn't smooth (it turns out $\Omega_X^1$
	isn't a vector bundle, which is often the problem).
	\item Let $f\in k[x]$ be a \term{separable} polynomial, meaning it has no repeated roots over the algebraic
	closure $\overline k$ of $k$, and consider the scheme $X = \Spec k[x,y]/(y^2 - f(x))$.\footnote{If $\deg f \ge
	5$, this is called a \term{hyperelliptic curve}.} From dimension theory, it's clear this is a curve (i.e.\
	$1$-dimensional); we'll show it's smooth.

	This time, in the resolution of $\Omega_X^1$
	\begin{equation}
		\xymatrix{
			A\ar[r]^-f & A^{\oplus 2}\ar[r]^-g & \Omega_X^1\ar[r] & 0,
		}
	\end{equation}
	$f(1) = \d(y^2-f(x)) = 2y\ud y - f'(x)\ud x$, so if we have a field $K$ and $x,y\in K$ satisfying $y^2 = f(x)$,
	then if $2y\ud y = f'(x)\ud x$, then $y = 0$ and $f'(x) = 0$. Then $x$ is a root of $f$ and $f'$, but since we
	assumed $f$ is separable, this cannot happen. Therefore $\Omega_X^1$ is a vector bundle.

	The picture is that restricting the projection $\A^2\to\A^1$ sending $(x,y)\mapsto x$ to $X$ defines a map
	whose fiber at $x\in\A^1$ is the square roots to $f(x)$ if $f(x)\ne 0$.
	\item Our last example is an important pathology to be aware of. Suppose $k$ has characteristic $p > 0$, and
	suppose $\lambda\in k$ is not a $p^{\mathrm{th}}$ power (in particular, $k$ is infinite; a typical example is
	$k = \F_p(\lambda)$, the field of rational functions in $\lambda$). Let $k'\coloneqq k[t]/(t^p-\lambda)$,
	adjoining a $p^{\mathrm{th}}$ root of $\lambda$; $t^p-\lambda$ is irreducible, so this is a field.

	Then $\Spec k'$ is zero-dimensional over $k$, but $\Omega_{\Spec k'}^1\ne 0$, so this point is not smooth,
	which is weird. This is because $\Omega_{\Spec k'}^1 = k'\cdot\d t/\d(t^p-\lambda)$, but $\d(t^p-\lambda) =
	pt^{p-1} = 0$, so $\Omega_{\Spec k'}^1$ is one-dimensional.

	$\Spec k'$ is, of course, smooth over itself, i.e.\ as a $k'$-scheme; smoothness is relative. The related
	notion of regularity is intrinsic, but smoothness is always with respect to a base. Said a different way,
	smoothness is a property of morphisms.
	\qedhere
\end{comp}


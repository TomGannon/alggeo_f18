\begin{quote}\textit{
	``This lemma existed 100 years ago, and will exist 100 years following. We're not a part of it.''
}\end{quote}

Today we'll classify smooth, zero-dimensional varieties over a field.
\begin{thm}
\label{0sch}
Let $X$ be a smooth, zero-dimensional finite-type scheme over a field $k$. Then
\begin{equation}
	X\cong \prod_{i=1}^n k_i,
\end{equation}
where each $k\inj k_i$ is a separable field extension.
\end{thm}
\begin{rem}
Recall that a field extension $k\inj k'$ is \term{separable} if for all $x\in k'$, the minimal polynomial of $x$
over $k$ has no repeated roots.
\end{rem}
First, one ingredient we'll need in the proof, and which will also be useful later.
\begin{defn}
Let $k$ be a field and $V$ be a vector space over $k$. The \term{split square-zero extension} associated to this
data is the commutative $k$-algebra $A = k\oplus V$ with the multiplication
\begin{equation}
	(\lambda, v)\cdot (\mu, w) \coloneqq (\lambda+\mu, \mu\cdot v + \lambda\cdot w).
\end{equation}
\end{defn}
In particular, $(0, v)\cdot (0,w) = 0$ and $(\lambda, 0)\cdot (0,v) = (0, \lambda v)$. If $V$ is one-dimensional,
this recovers the dual numbers; you can think of split square-zero extensions as generalizations of the dual
numbers.
\begin{proof}[Proof of \cref{0sch}]
We can assume $X = \Spec A$ is affine. Since $X$ is zero-dimensional, $A$ is Artinian; from the theory of Artinian
rings, $A$ is a product of Artinian local rings; since $X$ is finite type, this is a finite product. That is,
\begin{equation}
	A = \prod_{i=1}^n A_i,
\end{equation}
where $A_i$ is an Artinian local ring. Therefore we can reduce to the case where $A$ itself is an Artinian local
ring, with maximal ideal $\m$.

For the next step, we assume $A = k'$ is a finite separable extension of $k$; we'll show that $X = \Spec k'$ is
smooth, which in this setting is equivalent to showing that $\Omega_{X/k}^1 = 0$. $\Omega_{X/k}^1$ is generated by
the elements $\d x$ for $x\in k'$; given such an $x$, let $f_x(t)\in k[t]$ denote the minimal polynomial of $x$
over $k$. Then, $f(x) = 0$, so
\begin{equation}
	\d f(x) = f'(x)\ud x = 0.
\end{equation}
Since $f'(x)\ne 0$ by separability, then $\d x = 0$.

Now let $A = k\oplus V$ be a split square-zero extension. The projection map $A\to k$ is a ring map. The other
projection map $A\to V$ is a derivation, which is a quick thing to check. Therefore, in particular, $A$ is a field
if and only if $V = 0$.

Next, we'll show that if $A$ is an Artinian local $k$-algebra with nonzero maximal ideal $\m$ such that $\m^2 = 0$
and with a separable residue field $k'\coloneqq A/\m$, then $\Omega_{\Spec A/k}^1 \ne 0$. In this setting, $\m$
acts trivially on itself, hence the $A$-module structure on $\m$ passes to a $k'$-vector space structure.

We claim that $A$ is isomorphic to the split square-zero extension $k'\oplus\m$; then this step will follow from
the previous step. We'll show this by showing that the projection $\pi\colon A\to A/\m = k'$ splits canonically as
an algebra map. Specifically, if $x\in k'$, let $f(t)\in k[t]$ be its minimal polynomial and $\widetilde x\in A$ be
a lift of $x$ across $\pi$. Then, there's a unique $\sigma(x)\in A$ such that $f(\sigma(x)) = 0$ and
$\pi(\sigma(x)) = 0$.

If $v\in\m$, then
\begin{equation}
	f(\widetilde x+v) = f(\widetilde x) + f'(x)\cdot v,
\end{equation}
which you can prove by reducing to the case where $f(t) = t^n$ and check directly using the binomial theorem and
the face that $v^2 = 0$. Therefore $\pi(f(\widetilde x)) = f(x) = 0$, so $f(\widetilde x)\in\m$. Therefore
\begin{equation}
	v = -\frac{1}{\pi(f'(\widetilde x))} \cdot f(x)\in\m.
\end{equation}
Because $f$ is separable, $\pi(f'(\widetilde x)) \ne 0$.

So we've shown that if $A = k'\oplus\m$ is a split square-zero extension, then $\Omega_{\Spec A/k}^1\ne 0$; next
we'll show that if $A$ is a local $k$-algebra (not a field) with spearable residue field, then $\Omega_{\Spec
A/k}^1\ne 0$. Now assume that $A$ is a local ring with nonzero maximal ideal $\m$ with separable residue field but
not necessarily assuming $\m^2 = 0$. Note that the ring map $\pi\colon A \to A/\m^2$ is a surjection, which implies
that we have a closed embedding $i\colon \Spec A/\m^2 \to \Spec A$. This closed embedding yields a surjection
$i^*\Omega_{\Spec A/k}^1 \to \Omega_{(\Spec A/\m^2)/k}^1$, which we argued last class, and because
$i^*\Omega_{\Spec A/k}^1$ surjects onto something nonzero, $i^*\Omega_{\Spec A/k}^1$ is nonzero, and thus
$\Omega_{\Spec A/k}^1$ is nonzero as well. Next class, we will discuss the inseparable case.
\end{proof}

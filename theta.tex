Today we're in the business of proving \cref{totsp}.
\begin{lem}
\label{QCohalgepi}
Let $X$ be a scheme and $\sA,\sB\in\QCoh(X)$ be commutative algebras together with a map $f\colon\sA\to\sB$ of
algebras. Then $\Spec_X(\sB)\to\Spec_X(\sA)$ is a closed embedding iff $f$ is an epimorphism.
\end{lem}
This is a relative version of the definition of closed embeddings of affine schemes we gave awhile ago.

We also need to define the what taking the total space $\Theta$ does to morphisms. Ultimately this is because
$(\bl)^\vee$, $\Sym_{\sO_X}$, and $\Spec_X$ are all functors, so we know what they do on morphisms; two are
contravariant and one is covariant, so we get a covariant functor.

In part~\eqref{thetaclosed} of \cref{totsp}, \cref{QCohalgepi} tells us this is equivalent to
$\Sym_{\sO_X}\sE^\vee\to\Sym_{\sO_X}\sL^\vee$ is an epimorphism. This is a $\Z$-graded sheaf, and it suffices to
show this for $\Sym_{\sO_X}^d$ for each $d$.
\begin{ex}
Show that if $\sE^\vee\to\sL^\vee$, then $\Sym^d\sE^\vee\to\Sym^d\sL^\vee$ is too.
\end{ex}
\begin{proof}[Proof of \cref{totsp}, $\eqref{cokerepi} \iff \eqref{genuniti}$]
We know condition~\eqref{cokerepi} is equivalent to $\coker(i^\vee) = 0$. This can be checked on an open cover
$\fU$, such as an affine open cover which trivializes $\sE$ and $\sL$, as in~\eqref{genuniti}. In this case, for
each $U$ in $\fU$, $i|_U\colon\sO_U\to\sO_U^{\oplus r}$ is determined by a row vector $(f_1,\dotsc,f_r)^{\mathrm
T}$, and $i^\vee|_U\colon\sO_U^{\oplus r}\to\sO_U$ is the column vector $(f_1,\dotsc,f_r)$. Its image is the ideal
generated by $(f_1,\dotsc,f_r)$, so the cokernel is $0$ iff $(f_1,\dotsc,f_r) = \Fun(U)$.
\end{proof}
\begin{proof}[Proof of \cref{totsp}, $\eqref{genuniti} \implies \eqref{cokervb}$]
We can again assume without loss of generality that $U = \Spec B$ is affine, and now we have maps
\begin{equation}
\xymatrix{
	B\ar[r]^-{\begin{bsmallmatrix}f_1\\\vdots\\f_r\end{bsmallmatrix}} & B^{\oplus r}\ar[r]^{[g_1,\dotsc,g_r]} &B,
}
\end{equation}
where $\sum f_ig_i = 1$. Since the composition is the identity, the map is split, and so the cokernel is a direct
summand, in particular is a direct summand of a free module, and must be free.
\end{proof}
\begin{ex}
Conversely, show that in \cref{totsp}, $\eqref{cokervb}\implies\eqref{genuniti}$.
\end{ex}
Since~\eqref{Talpha} tautologically implies~\eqref{ptalpha}, it suffices to show $\eqref{ptalpha}
\implies\eqref{cokerepi}$ and $\eqref{cokerepi}\implies\eqref{Talpha}$; \eqref{thetaclosed} is addressed by one of
the exercises above, I think (\TODO: I probably just missed it.)

Since $\coker(i)$ is locally finitely generated, Nakayama's lemma applies: if $V$ denotes the complement of the
support of $\coker(i)$, then $V$ is open, and $S = \Spec A\to X$ factors through $V$ iff $\coker(i)|_S = 0$.
So~\eqref{cokerepi} is equivalent to $V = X$.
\begin{proof}[Proof of \cref{totsp}, $\eqref{ptalpha}\implies\eqref{cokerepi}$]
To show $V = X$, it suffices to show any map $x\colon \Spec k\to X$ factors through $V$, since $V$ is open. One can
show that $x^*$ commutes with cokernels: since $\Spec k$ is affine, it arises as a tensor product, and tensor
products are right exact.\footnote{More generally, $x^*$ is a left adjoint, even not on affines; this automatically
means it's right exact.} Therefore it suffices to show that $x^*(\coker(i^\vee)) = 0$. If $x^*(i^\vee)$ is nonzero,
then $x^*(\sE)\to x^*(\sL)$ is a nonzero map to a 1-dimensional vector space, hence surjective, and therefore the
cokernel is zero.
\end{proof}
\begin{proof}[Proof of \cref{totsp}, $\eqref{cokerepi} \implies\eqref{Talpha}$]
Let $x\colon T\to X$ be a map, where $T$ if affine. As before, $x^*$ is right exact, hence commutes with cokernels.
By assumption, $\coker(i^\vee) = 0$, so $x^*(i^\vee)\colon x^*(\sE^\vee)\to x^*(\sL^\vee)$ is an epimorphism. This
is the dual map to $x^*(i)$, so if $T\ne\varnothing$, this means $x^*(i)\colon x^*(\sL)\to x^*(\sE)$ is nonzero.
\end{proof}
These equivalent conditions are all examples of the nicest possible maps of vector bundles. It's good to have all
of these different perspectives partly because they provide flexibility in the nice case, but also because it will
be useful to know what happens when things go bad. When we study projective varieties, several of these conditions
will come up. For example, it will be useful for showing $\P^n$ is a scheme!

Today, we will discuss various finiteness hypotheses one can put on a scheme.
\begin{defn}
A scheme $X$ is \term{quasicompact} if it has a finite affine open cover.
\end{defn}
Quasicompactness refers to the usual definition of compactness with respect to the Zariski topology, but without
any Hausdorff condition. On the other hand, this has \emph{nothing} to do with compactness in the ``usual''
topological sense. For example, the line $\A^1$ is quasicompact. The analogue of compactness in the usual sense
(e.g.\ manifold topology) is known as \term{properness}.
\begin{exm}
$\P^n$ is quasicompact, since it has a cover by $n+1$ opens, as we saw on Friday.
\end{exm}
\begin{prop}
If $A$ is a commutative ring and $I\subset A$ is an ideal, then $\Spec A\setminus \Spec(A/I)$ is quasicompact if
$I$ is finitely generated.
\end{prop}
The converse is false.
\begin{exm}
Let $\A_k^\infty\coloneqq\Spec k[x_1,x_2,x_3,\dotsc]$. Then $\A_k^\infty\setminus 0$ is a scheme which is not
quasicompact.
\end{exm}
\begin{rem}
More generally, a morphism $f\colon X\to Y$ is said to be quasicompact if for all affine schemes $S$ and maps $S\to
Y$, $X\times_Y S$ is quasicompact. This reduces to the notion of quasicompactness for schemes when $Y = \Spec\Z$,
because a map to $\Spec\Z$ is no data at all.
\end{rem}
\begin{defn}
A scheme $X$ is \term{quasiseparated} if the intersection of two affine opens in $X$ is quasicompact.
\end{defn}
Equivalently, the diagonal map $\Delta\colon X\to X\times X$ is a quasicompact morphism. Therefore one may more
generally say a quasiseparated morphism $f\colon X\to Y$  is one for which the diagonal $\Delta_f\colon X\to
X\times_Y X$ is quasicompact.
\begin{exm}
The idea is that quasiseparated is kind of like the Hausdorff property in topology. As such, we can import the
standard counterexample, $\A^1$ with two origins. The idea is to take two copies of $\A^1$, say with coordinates
$t$ and $u$, and glue them together along $\A^1\setminus 0$ via the identification $t\leftrightarrow u$. (This is
different from $\P^1$, which is quasiseparated, where we identified $t\leftrightarrow u^{-1}$.)
\end{exm}
This next exercise is a 
\begin{ex}
Quasicompact, quasiseparated morphisms are exactly those where the pushforward of quasicoherent sheaves is
well-behaved. Specifically, given a map $f\colon X\to Y$ and a quasicoherent sheaf $\sF$ on $X$, we define its
pushforward $f_*\sF$ on $Y$ as follows: given an affine open subscheme $j\colon U \inj Y$, we specify
\[j^*f_*\sF \coloneqq\Gamma(X\times_Y U, \sF|_{X\times_Y U}).\]
Show that this is quasicoherent if $f$ is quasicompact and quasiseparated. (\TODO: converse?)
\end{ex}
Even when we restrict to quasicompact, quasiseparated things, we're still looking at extremely general objects,
considerably moreso than are studied in classical algebraic geometry. So here are some more finiteness hypotheses.
\begin{defn}
Let $f\colon X\to Y$ be a map of schemes.
\begin{enumerate}
	\item Suppose $Y = \Spec A$. Then $f$ is \term{locally of finite type} (LFT) if for all affine opens $\Spec
	B\subset X$, the induced map of rings $A\to B$ makes $B$ into a finitely generated algebra, i.e.\ $B\cong
	A[x_1,\dotsc,x_n]/I$ for some ideal $I\subset A[x_1,\dotsc,x_n]$.
	\item If in addition $X$ is quasicompact, then $f$ is called \term{finite type}.
	\item For general $Y$, $f$ is \term{locally finite type} (resp.\ \term{finite type}) if for all affine opens
	$V\subseteq Y$, the map $X\times_Y V\to V$ is locally finite type (resp.\ finite type).
\end{enumerate}
\end{defn}
These properties roughly mean that you're covered by finite type algebras. It's not hard to prove that when $Y$ is
affine, these definitions coincide.
\begin{thm}[Hilbert's basis theorem]
\label{hilbbasis}
Suppose $A$ is a Noetherian ring amd $B$ is a finitely generated $A$-algebra. Then there is a fiber product square
\[\xymatrix{
	\Spec B\ar[r]\ar[d] & \A^n_A\ar[d]\\
	0\ar[r] & \A_A^m.
}\]
Here $\A_A^k\coloneqq\Spec A[x_1,\dotsc,x_k]$.
\end{thm}
This will be highly noncanonical.
\begin{rem}
This is telling us that $\Spec B$ is the zero locus of $m$ polynomials in $n$ variables with coefficients in $A$,
or that more generally, a finite type Noetherian scheme is exactly one which locally admits such a description.
Since this was one of our motivations for studying algebraic geometry from the beginning, this is an excellent
hypothesis to have.
\end{rem}
Though you've probably already seen the proof if you know what a Noetherian ring is, it's still good to go over.
\begin{defn}
Let $V$ be an abelian group. A \term{filtration} on $V$ is a sequence
\begin{equation}
	F_0V\subseteq F_1V\subseteq\dotsb\subseteq V,
\end{equation}
such that
\begin{equation}
	\bigcup_{n\ge 0} F_nV = V.
\end{equation}
The \term{associated graded} is a graded abelian group $\gr_\bullet(V)\coloneqq \bigoplus_{n\ge 0}\mathrm{gr}_n V$,
where $\mathrm{gr}_nV \coloneqq F_nV/F_{n-1}V$, and we declare $F_{-1}V\coloneqq 0$.
\end{defn}
Filtrations are more general than gradings, but are really nice to have, and can make some arguments a lot cleaner.
\begin{defn}
If $(V, F_n)$ and $(W, F_n')$ are filtrations, a morphism of filtered abelian groups is a map $f\colon V\to W$ such
that $f(F_nV)\subseteq F_n'W$.
\end{defn}
\begin{ex}
\label{grsurj}
Show that if $\gr_\bullet(f)\colon\gr_\bullet(V)\to\gr_\bullet(W)$ is injective (resp.\ surjective, resp.\
bijective), then $f$ is injective (resp.\ surjective, resp.\ bijective).
\end{ex}
This is a major tool in working with filtrations, especially in the (common) case where the filtered objects are
complicated, but their associated gradeds are simpler.
\begin{defn}
A \term{filtered algebra} is an algebra $A$ filtered as an abelian group such that multiplication carries
$F_nA\times F_mA\subseteq F_{n+m}A$.
\end{defn}
\begin{exm}
The algebra $A = k[x]$ is filtered by degree: we let $F_nA$ denote the polynomials of degree at most $n$.
\end{exm}
If $A$ is a filtered abelian group, $F_nA$ is also an abelian group, but if $A$ is an algebra, $F_nA$ is generally
not a subalgebra, as in the above example.

If $A$ is a filtered algebra, we can make sense of the notion of filtered $A$-modules, where the filtrations given
by the $A$-action and the module are compatible in the least surprising way.
\begin{lem}
If $M$ is a filtered $A$-module and $\gr_\bullet(M)$ is a finitely generated $\gr_\bullet(A)$-module, then $M$ is a
finitely generated $A$-module.
\end{lem}
\begin{proof}
Let $\overline x_1,\dotsc,\overline x_n\in\gr_\bullet M$ be a generating set. We can assume they're
\term{homogeneous}, i.e.\ each $\overline x_i$ lives in some $\gr_{k_i}M$. Lift $\overline x_i$ to some $x_i\in
F_{i_k}M$; then the map $\vp\colon A^{\oplus n}\to M$ sending the standard basis element $e_i\mapsto x_i$ is
surjective after passing to the associated graded, hence by \cref{grsurj} is surjective, and therefore
$\set{x_1,\dotsc,x_n}$ generates $M$.
\end{proof}
\begin{proof}[Proof of \cref{hilbbasis}]
By induction, it suffices to show that if $A$ is Noetherian, then $A[x]$ is too. Filter $A[x]$ by degree, and if
$I\subseteq A$ is an ideal, let $F_nI\coloneqq I\cap F_nA[x]$.

We claim $\gr_\bullet I$ is finitely generated over $\gr_\bullet A[x]$. To see this, note that the
multiplication-by-$x$ map $\gr_i I \to\gr_{i + 1}I$ is an injection for all $i$, and furthermore its image in $A$
corresponds to the inclusion of an ideal. Therefore, by Noetherianness of $A$, this chain must stabilize at some $N
\in \N$, and therefore $\gr_\bullet I$ is generated by $\bigoplus_{i = 0}^N \gr_iI$. Because $A$ is Noetherian,
each $\gr_iI$ is a finitely generated $A$-module as well, showing the claim.  
\end{proof}

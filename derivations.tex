\begin{quote}\textit{
	``It's not a field, but it's psychologically a field.''
}\end{quote}

Today we're going to talk about differentials and derivations, which are pretty important. For this lecture, $X =
\Spec A$ is an affine scheme over a field $k$.
\begin{defn}
If $M$ is an $A$-module, a \term{derivation} $\delta\colon A\to M$ is an $A$-linear map satisfying the Leibniz rule
\[\delta(fg) = f\delta(g) + g\delta(f).\]
The set of derivations from $A$ to $M$ is denoted $\Der_A(A, M)$; it is naturally an $A$-module.

A \term{vector field} is a derivation $\delta\colon A\to A$.
\end{defn}
In differential geometry, a vector field gives you a way to differentiate functions.

Derivations are corepresented by a particular $A$-module (i.e.\ quasicoherent sheaf on $X$) $\Omega_X^1$: that is,
it's equipped with a derivation $\d\colon\sO_X\to\Omega_X^1$ such that for all $A$-modules $M$, restriction along
$\d$ defines an $A$-linear isomorphism
\begin{equation}
	\Hom_A(\Omega_X^1, M)\overset\cong\longrightarrow \Der_A(A,M).
\end{equation}
The proof is a contruction: let $\Omega_X^1$ be generated as an $A$-module by elements $\set{\d f\mid f\in A}$
with relations
\begin{subequations}
\begin{alignat}{2}
	\d(fg) &= f\ud g + g\ud f\qquad &\text{ for all }& f,g \in A\\
	\d(\lambda f) &= \lambda\ud f &\text{for all } &\lambda \in k.
\end{alignat}
\end{subequations}
\begin{lem}
If $f\in A$ and $n\ge 1$, then $\d(f^n) = nf^{n-1}\ud f\in\Omega_X^1$.
\end{lem}
\begin{proof}
Induct on $n$: it's clear for $n = 1$, and assuming it for $n$, it follows for $n+1$ using the Leibniz rule on
$f^{n+1} = (f^n)(f)$.
\end{proof}
As a corollary, $\d(1) = 0$, as $\d(1^n) = \d(1)$.
\begin{exm}
\label{A1der}
For $X = \A_k^1 = \Spec k[t]$, we have $\d t\in\Omega_{\A_k^1}^1$. We claim $\Omega_{\A_k^1}^1$ is freely generated
by $\d t$, i.e.\ $\Omega_{\A_k^1}^1 \cong \sO_{\A_k^1}\cdot\d t$.

The proof is that, given a $k[t]$-module $M$ and a derivation $\delta\colon k[t]\to M$, if $f = \sum a_t^i\in
k[t]$, then in $M$,
\begin{equation}
	\delta(f) = \sum_{i\ge 1} a_i it^{i-1} \delta(t)\in M.
\end{equation}
so it's spanned by $\d t$. Conversely, given an element $\delta(t)\in M$, there's a unique derivation $\delta\colon
k[t]\to M$ sending $t\mapsto\delta(t)$, by the universal property, so $k[t]$-linear maps $\Omega_X^1\to M$ are
uniquely determined by where they send $\d t$.
\end{exm}
So if you boil off the abstraction, all you need is to know the derivative of a polynomial. Hopefully this is
reassuring.
\begin{exm}
Now take $X = \A_k^n = \Spec k[t_1,\dotsc,t_n]$. Now $\Omega_{\A_k^n}^1$ is a free $k[t_1,\dotsc,t_n]$-module of
rank $n$, with a basis $\d t_1,\dotsc, \d t_n$, and
\begin{equation}
	\d f = \sum_{i=1}^n \pfr{f}{t_i}\ud t_i.
\end{equation}
The proof is essentially the same as for \cref{A1der}: once you know where $t_1,\dotsc,t_n$ go, everything else is
forced by linearity and the Leibniz rule.
\end{exm}
For a general finite type affine $X = \Spec A$, there's a general algorithm to compute $\Omega_X^1$: first let
$f_1,\dotsc,f_r$ generate $A$ as a $k$-algebra, and write
\begin{equation}
	A\cong k[t_1,\dotsc,t_r]/(g_1,\dotsc,g_s)
\end{equation}
for some $g_1,\dotsc,g_s$ encoding the relations between the $f_i$. Then $\Omega_X^1$ is generated by $\d
f_1,\dotsc,\d f_r$ with relations $\d g_i|_X = 0$ for $i = 1,\dotsc,s$.

What's going on here? Well a map $\vp\colon Y\to X$ of affine $k$-schemes induces a map of $A$-modules $\d\vp\colon
\Omega_X^1\to\vp_*\Omega_Y^1$: take the universal differential $\d_Y\colon\sO_Y\to\Omega_Y^1$ and push it forward
to $X$: $\vp_*\d_Y\colon \vp_*\sO_Y\to\vp_*\Omega_Y^1$. Then precompose with $\vp^*\colon \sO_X\to\sO_Y$ (pullback
of functions); this is a differential $\sO_X\to\vp_*\Omega_Y^1$, hence corresponds uniquely to an $A$-module map
$\Omega_X^1\to\vp_*\Omega_Y^1$. Somewhat more explicitly, the characteristic formula is
\begin{equation}
	\d\vp(f\ud g) = (f\vp)\ud (g\vp).
\end{equation}
\begin{rem}
Often, $\Omega_X^1$ is denoted $\Omega_{X/k}^1$: $k$-linearity is made more explicit. For example, nothing we've
done so far requires $k$ to be a field, so we could work with affine schemes over a ring $B$ and study the module
of differentials $\Omega_{X/B}^1$.
\end{rem}
Suppose $i\colon X\inj Y$ is a closed embedding of affine schemes over $k$.  Then one can show that the induced map
$i^*\Omega_Y^1\to \Omega_X^1$ is surjective; if $I$ denotes its kernel, then $I\subseteq\sO_Y$ is an ideal.
\begin{exm}
$\Omega_X^1$ isn't always free; for example, suppose $k$ has characteristic zero and $X = \Spec(k[t]/(t^n))$. Using
the above algorithm, one can show $\Omega_X^1$ is torsion.
\end{exm}
